\documentclass[11pt,a4paper]{article}

\usepackage[left=2cm,text={17cm,24cm},top=3cm]{geometry}
\usepackage[english]{babel}
\usepackage[utf8]{inputenc}
\usepackage[T1]{fontenc}

\usepackage{url}
\usepackage{tikz}
\usepackage{float}
\usepackage{xcolor}
\usepackage{siunitx}
\usepackage{amsmath}
\usepackage{accents}
\usepackage{comment}
\usepackage{listings}
\usepackage{csquotes}
\usepackage{hyperref}
\usepackage{textcomp}
\usepackage{amsfonts}
\usepackage{breakurl}
\usepackage{etoolbox}
\usepackage{graphicx}
\usepackage{multicol}
\usepackage{multirow}
\usepackage{indentfirst}
\usepackage{supertabular}
\usepackage[titles]{tocloft}
\usepackage{svg}
\usepackage{amsmath}
\usepackage[final]{pdfpages}

\def\UrlBreaks{\do\/\do-} % URL breaking characters

\newcommand{\red}[1]{\textcolor{red}{#1}} % \red{text in red}
\newcommand{\blue}[1]{\textcolor{blue}{#1}} % \blue{text in blue}
\newcommand{\TODO}{\textbf{\textcolor{red}{TODO}}} % red bold TODO
\newcommand{\tilda}{\raisebox{0.5ex}{\texttildelow}} % command \tilda for '~' character

\renewcommand{\cftdot}{}

\setlength\parindent{0pt} % do NOT indent
\graphicspath{{img/}} % path to images

\patchcmd{\thebibliography}{\section*{\refname}}{}{}{}

\begin{document}

\begin{titlepage}

    \begin{center}
        % FIX: lines must end with '%', if not then it will result in an incorrect centering
        \vfill {%
            \Huge{%
                \textsc{%
                    Faculty of Informatics\\[3mm]%
                    Masaryk University%
                }%
            }%
        }%

        \hfill\\[15mm]

        \begin{figure}[!h]
            \centering
            \includegraphics[scale=2]{muni-lg-cmyk.pdf}
        \end{figure}

        \hfill\\[10mm]

        \Huge{
            \textbf{
                Process analysis
            }
        }

        \hfill\\[-10mm]

        \huge{
            \textbf{
                Team project for PV207
            }
        }

        \hfill\\[10mm]

        \LARGE{
            \textbf{
                Food delivery service - FoodPack
            }
        }
        \vfill

    \end{center}

        \Large{
            \textbf{Team:}\\Tadeáš Pavlík (Teamleader),\\Jiří Čechák (Business analyst),\\Tomáš Došlík (Process analyst),\\Václav Stehlík (BPM/SOA developer)\hfill \today
        }

\end{titlepage}

\setlength{\parskip}{0pt}
    \hypersetup{hidelinks}\tableofcontents
\setlength{\parskip}{0pt}

\newpage

\section{Organisation overview}

FoodPack is a new Czech startup that wants to change the world of cooking and food preparation. FoodPack wants to offer its users the possibility of quick and carefree preparation of dishes without the need for lengthy and complicated purchase of ingredients. The offering consists of a continually expanding range of carefully crafted recipes and distributes all the necessary ingredients of the highest quality for food preparation to your home.

\subsection{Vision}

Our vision is to make sure that everyone cooks effectively, healthy and from fresh ingredients.

\subsection{Mission}

Our mission is to create and provide healthy food recipes to all people who want to eat healthy and to create a distribution network to quickly deliver packages of fresh and healthy ingredients for efficient cooking.

\subsection{Goals and objectives}
\begin{itemize}
    \item Goal: Provide and distribute packages of fresh ingredients for home cooking of food according to our recipes.
    \begin{itemize}
        \item Objective: Create and develop at least 3 top recipes of the month every month.
        \item Objective: Ensure efficient and timely delivery.
        \item Objective: Ensure storage of ingredients.
        \item Objective: Ensure effective preparation of orders.
    \end{itemize}
    \item Goal: Ensure the best local suppliers of all used and needed fresh ingredients.
    \begin{itemize}
        \item Objective: Explore local market of suppliers and identify their offer of ingredients.
        \item Objective: Verify the suitability and safety of ingredients offered from suppliers.
    \end{itemize}
    \item Goal: Increase awareness about our company and its services.
    \begin{itemize}
        \item Objective: Develop and integrate plausible advertising strategy.
        \item Objective: Improve propagation via the internet, social networks and TV.
        \item Objective: Plan and take part in food events.
    \end{itemize}
    \item Goal: Expand to other countries.
    \begin{itemize}
        \item Objective: Analyse market of other countries.
        \item Objective: Establish new distribution centers in other countries.
        \item Objective: Find new drivers for new distribution centers.
    \end{itemize}
\end{itemize}

\subsection{Measurement indicators}

\begin{itemize}
    \item Objective: Create and develop at least 3 top recipes of the month every month.
    \begin{itemize}
        \item KPI: Daily counts of views, clicks, orders.
        \item KRI: Total number of orders in the first two months.
    \end{itemize}
    \item Objective: Ensure efficient and timely delivery.
    \begin{itemize}
        \item KPI: Order response speed.
        \item KRI: Success rate of deliveries per month.
    \end{itemize}
    \item Objective: Ensure storage of ingredients.
    \begin{itemize}
        \item KPI: 
        \item KRI: 
    \end{itemize}
    \item Objective: Ensure effective preparation of orders.
    \begin{itemize}
        \item KPI: 
        \item KRI: 
    \end{itemize}
    \item Objective: Explore local market of suppliers and identify their offer of ingredients.
    \begin{itemize}
        \item KPI: Negotiated offers from suppliers.
        \item KRI: Total successfully closed agreements with negotiated suppliers
    \end{itemize}
    \item Objective: Verify the suitability and safety of ingredients offered from suppliers.
    \begin{itemize}
        \item KPI: 
        \item KRI: 
    \end{itemize}
    \item Objective: Develop and integrate plausible advertising strategy.
    \begin{itemize}
        \item KPI: 
        \item KRI: 
    \end{itemize}
    \item Objective: Improve propagation via the internet, social networks and TV.
    \begin{itemize}
        \item KPI: 
        \item KRI: 
    \end{itemize}
    \item Objective: Plan and take part in food events.
    \begin{itemize}
        \item KPI: 
        \item KRI: 
    \end{itemize}
    \item Objective: Analyse market of other countries.
    \begin{itemize}
        \item KPI: 
        \item KRI: 
    \end{itemize}
    \item Objective: Establish new distribution centers in other countries.
    \begin{itemize}
        \item KPI: 
        \item KRI: 
    \end{itemize}
    \item Objective: Find new drivers for new distribution centers.
    \begin{itemize}
        \item KPI: 
        \item KRI: 
    \end{itemize}
\end{itemize}

\subsection{Organisation structure}
\subsubsection{Roles and responsibilities}

\begin{itemize}
    \item \textbf{Chef} - Creates new recipes from proposed recipe ideas using specific ingredients. He is an external employee.
    \item \textbf{Editor} - Makes correction of recipes, adds description for recipes, prepares recipes for publishing on the web.
    \item \textbf{Photographer} - Takes photos of ingredients and prepared food, edits photos.
    \item \textbf{Logistics Manager} - Plans new routes for drivers and distribution centers, manages delivery of packages.
    \item \textbf{CEO} - Makes money, creates company strategy, planning, oversees and motivates employees.
    \item \textbf{Driver} - Safely and quickly delivers orders to customers.
    \item \textbf{Warehouse Worker} - Takes care of ingredients and items in distribution center warehouse, packs ingredients into packages when order comes, negotiates with suppliers.
    \item \textbf{Distribution Center Manager} - Makes sure shipped orders are delivered in the most effective way possible.
    \item \textbf{Procurement Officer} - Orders goods from external suppliers so that the warehouses are always stocked.
    \item \textbf{Order Manager} - Overlooks and is responsible for the order processing.
\end{itemize}

\subsubsection{Departments}

\begin{itemize}
    \item \textbf{Research & Development} - Prepares new recipes, recipe materials, manages chefs.
    \item \textbf{Testing department} - Performs feasibility check of chef’s recipe. Feasibility check includes preparation testing, taste testing and cost/pricing calculation.
    \item \textbf{Marketing department} - Presents the company to the world, creates and manages marketing campaigns and creates customer reach.
    \item \textbf{Market research department} - Provides user feedback, trend analysis and estimates demand.
    \item \textbf{Logistic department} - Responsible for planning of routes, assigning routes to drivers and distribution centers, delivering of packages.
    \item \textbf{Human Resources department} - Finding and recruiting job applicants, and administering employee-benefit programs. Manages benefits, recruitment and firing of employees.
    \item \textbf{IT department} - Manages web, database, online shop and blogs and is in charge of company security, solves issues and errors concerning IT in company.
    \item \textbf{Packaging department} - Responsible for preparing packages, maintaining the ingredients and contents of warehouse, negotiation with suppliers.
    \item \textbf{Procurement} - Acquiring goods and services from an external sources.
\end{itemize}

\section{Processes}

\begin{itemize}
    \item Find experienced chef. (Objective: Create and develop at least 3 top recipes of the month every month.)
    \item Create a new recipe. (Objective: Create and develop at least 3 top recipes of the month every month.)
    \item Preparation of recipe materials. (Objective: Create and develop at least 3 top recipes of the month every month.)
    \item Route planning. (Objective: Ensure efficient and timely delivery.)
    \item Find suppliers within a particular distribution center location. (Objective: Explore local market of suppliers and identify their offer of ingredients.)
    \item Order processing. (Objective: Ensure effective preparation of orders.)
    \item Customer support. (Objective: Ensure effective preparation of orders.)
    \item Marketing. (Objective: Improve propagation via the internet, social networks and TV.)
    \item Create statistics report for CEO and management. (Objective: Ensure efficient and timely delivery.)
\end{itemize}

\newpage

\subsection{Process "Preparation of recipe materials"}

\paragraph{Description}

This process is situated after Research \& Development with help of chef creates a recipe. It is needed to edit the recipe, add more description to it and create photos for the recipe. In the end all these parts will make it up to materials for recipe. After that recipe will be published if we already have needed ingredients. If not recipe will be stored without publishing and Research \& Development will be notified for which recipe materials we are missing ingredients so that the ingredients could be achieved.

\paragraph{Indicators}

\begin{itemize}
    \item \textbf{KPI}: Daily counts of views, clicks, orders. (At least 500 views, 150 clicks, 50 orders)
    \item \textbf{KRI}: Total number of orders in the first two months. (At least 3000 orders)
\end{itemize}

\paragraph{Roles}

\begin{itemize}
    \item \textbf{Editor} - Performs correction of recipe received from Research \& Development, creates description for it, send final text to system.
    \item \textbf{System} - Requests materials from Editor and Photographer, stores them into store, publishes recipe if ingredients are available, notices Research \& Development if not.
    \item \textbf{Photographer} - Request and receives prepared meal from Chef, takes photo it, edits photos and sends final photos to system.
    \item \textbf{Chef} - Receives request for preparation of a meal, serves prepared meal to photographer.
    \item \textbf{Research \& Development} - Requests materials for recipe and receives notification if final recipe can not be published.
\end{itemize}

\paragraph{Data objects}

\begin{itemize}
    \item \textbf{Recipe store} - Storage for created and finalized recipes.
    \item \textbf{Ingredient store} - Storage for information about available ingredients.
\end{itemize}

\newpage

\includepdf[landscape=true]{Preparation of recipe materials.pdf}

\subsection{Process "Find suppliers within a particular distribution center location"}

\paragraph{Description}

This process is situated after manager of distribution center requests for searching suppliers. It is needed to look at which ingredients are requested for recipes. Then some worker has to check warehouse of distribution center which ingredients are missing a ask system to find suppliers for them. System will find supplier by using external searching module. If its response is too long then it's not working and we can't find any suppliers. Otherwise list of suitable suppliers will be received. The worker will have to negotiate with suppliers and then decide if negotiated prices are good for the company. After that manager is informed about negotiated suppliers if there are some. Otherwise he will be informed about reasons of failure.

\paragraph{Indicators}

\begin{itemize}
    \item \textbf{KPI}: Negotiated offers from suppliers. (At least 1 per day)
    \item \textbf{KRI}: Total successfully closed agreements with negotiated suppliers. (At least 10 per month)
\end{itemize}

\paragraph{Roles}

\begin{itemize}
    \item \textbf{Warehouse Worker} - Looks after warehouse and items in it, says what is missing to system, negotiates with suppliers and decides which prices are acceptable.
    \item \textbf{System} - Works with storage, checks for ingredients, communicates with searching module, generates report for manager.
    \item \textbf{Distribution Center Manager} - Starts process by request for searching of suppliers and gets notified about result.
    \item \textbf{IT department} - Gets notified if there are issues with store.
    \item \textbf{Searching module} - Searches for nearby suppliers based on criteria formulated by Warehouse Worker, provides suppliers suiting criteria to system.
    \item \textbf{Supplier} - Negotiates with Warehouse Worker about prices of ingredients.
    
\end{itemize}

\paragraph{Data objects}

\begin{itemize}
    \item \textbf{Store} - Storage for ingredients and recipes.
    \item \textbf{Supplier store} - Storage for suppliers who negotiated with Warehouse Worker.
\end{itemize}

\newpage

\includepdf[landscape=true]{Find suppliers within a particular distribution center location.pdf}

\section{Implementation}

[text]

\subsection{Used platform and software}

[text]

\subsection{Implemented services}

[text]

\begin{enumerate}
    \item Service name
    \begin{itemize}
        \item text
    \end{itemize}
\end{enumerate}

\subsection{Implemented processes}

\begin{enumerate}
    \item Process name
    \begin{itemize}
        \item text
    \end{itemize}
\end{enumerate}

\subsection{Screenshots}

[3-5 important screenshots from testing of the implementation]

\section{Teamwork and tasks}

\subsection{Tadeáš Pavlík (Teamleader)}

\begin{itemize}
    \item task 1
\end{itemize}

\subsection{Jiří Čechák (Business analyst)}

\begin{itemize}
    \item task 1
\end{itemize}

\subsection{Tomáš Došlík (Process analyst)}

\begin{itemize}
    \item task 1
\end{itemize}

\subsection{Václav Stehlík (BPM/SOA developer)}

\begin{itemize}
    \item task 1
\end{itemize}

\end{document}
