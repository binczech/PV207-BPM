\documentclass[11pt,a4paper]{article}

\usepackage[left=2cm,text={17cm,24cm},top=3cm]{geometry}
\usepackage[czech]{babel}
\usepackage[utf8]{inputenc}
\usepackage[T1]{fontenc}

\usepackage{url}
\usepackage{tikz}
\usepackage{float}
\usepackage{xcolor}
\usepackage{siunitx}
\usepackage{amsmath}
\usepackage{accents}
\usepackage{comment}
\usepackage{listings}
\usepackage{csquotes}
\usepackage{hyperref}
\usepackage{textcomp}
\usepackage{amsfonts}
\usepackage{breakurl}
\usepackage{etoolbox}
\usepackage{graphicx}
\usepackage{multicol}
\usepackage{multirow}
\usepackage{indentfirst}
\usepackage{supertabular}
\usepackage[titles]{tocloft}

\def\UrlBreaks{\do\/\do-} % URL breaking characters

\newcommand{\red}[1]{\textcolor{red}{#1}} % \red{text in red}
\newcommand{\blue}[1]{\textcolor{blue}{#1}} % \blue{text in blue}
\newcommand{\TODO}{\textbf{\textcolor{red}{TODO}}} % red bold TODO
\newcommand{\tilda}{\raisebox{0.5ex}{\texttildelow}} % command \tilda for '~' character

\renewcommand{\cftdot}{}

\setlength\parindent{0pt} % do NOT indent
\graphicspath{{img/}} % path to images

\patchcmd{\thebibliography}{\section*{\refname}}{}{}{}

\begin{document}

\begin{titlepage}

    \begin{center}
        % FIX: lines must end with '%', if not then it will result in an incorrect centering
        \vfill {%
            \Huge{%
                \textsc{%
                    Fakulta informatiky\\[3mm]%
                    Masarykova univerzita%
                }%
            }%
        }%

        \hfill\\[15mm]

        \begin{figure}[!h]
            \centering
            \includegraphics[scale=2]{muni-lg-cmyk.pdf}
        \end{figure}

        \hfill\\[10mm]

        \Huge{
            \textbf{
                Process analysis
            }
        }

        \hfill\\[-10mm]

        \huge{
            \textbf{
                Týmový projekt pro PV207
            }
        }

        \hfill\\[10mm]

        \LARGE{
            \textbf{
                [Doména] - FoodPack
            }
        }
        \vfill

    \end{center}

        \Large{
            \textbf{Tým:}\\Tadeáš Pavlík (Teamleader),\\Jiří Čechák (Business analyst),\\Tomáš Došlík (Process analyst),\\Václav Stehlík (BPM/SOA developer)\hfill \today
        }

\end{titlepage}

\setlength{\parskip}{0pt}
    \hypersetup{hidelinks}\tableofcontents
\setlength{\parskip}{0pt}

\newpage

\section{Přehled organizace}

\subsection{Vize}

[Vložte vizi zde]

\subsection{Mise}

[Vložte misi zde]

\subsection{Cíle a objektivy}
\begin{itemize}
    \item Cíl: Poskytnout a distribuovat balíčky čerstvých surovin pro domácí přípravu našich receptů.
    \begin{itemize}
        \item Objektiv: Vytvořit a vyvinout každý měsíc alespoň 3 top recepty měsíce.
        \item Objektiv: Zajistit efektivní a časově přesný rozvoz.
    \end{itemize}
\end{itemize}

\subsection{Ukazatele měření}

\begin{itemize}
    \item Objektiv: Vytvořit a vyvinout každý měsíc alespoň 3 top recepty měsíce.
    \begin{itemize}
        \item KPI: Denní počet shlédnutí, prokliků, objednávek.
        \item KRI: Celkový počet objednávek za první dva měsíce.
    \end{itemize}
    \item Objektiv: Zajistit efektivní a časově přesný rozvoz.
    \begin{itemize}
        \item KPI: Rychlost reakce na objednávku.
        \item KRI: Úspěšné ku neúspěšných rozvozům za měsíc.
    \end{itemize}
\end{itemize}

\subsection{Struktura organizace}
\subsubsection{Role a odpovědnosti}

[text]

\subsubsection{Oddělení}

[text]

\section{Procesy}

\begin{itemize}
    \item Najít zkušeného kuchaře. (Vytvořit a vyvinout každý měsíc alespoň 3 top recepty měsíce.)
    \item Přidat nový recept. (Vytvořit a vyvinout každý měsíc alespoň 3 top recepty měsíce.)
    \item Příprava materiálů k receptům (fotky, ...). (Vytvořit a vyvinout každý měsíc alespoň 3 top recepty měsíce.)
    \item Plánování tras. (Zajistit efektivní a časově přesný rozvoz.)
\end{itemize}

\subsection{Proces \uv{Najít zkušeného kuchaře.}}

\paragraph{Popis}

[text]

\paragraph{Indikátory}

\begin{itemize}
    \item název indikátoru
    \begin{itemize}
        \item units [formula]
        \item desired value [minimal value]
    \end{itemize}
\end{itemize}

\paragraph{Role}

\begin{itemize}
    \item název role
    \begin{itemize}
        \item popis
    \end{itemize}
\end{itemize}

\paragraph{Datové objekty}

\begin{itemize}
    \item název datového objektu
    \begin{itemize}
        \item popis [states]
    \end{itemize}
\end{itemize}

\paragraph{Diagram}

[zde bude BPMN diagram]

\subsection{Proces \uv{Přidat nový recept.}}

\paragraph{Popis}

[text]

\paragraph{Indikátory}

\begin{itemize}
    \item název indikátoru
    \begin{itemize}
        \item units [formula]
        \item desired value [minimal value]
    \end{itemize}
\end{itemize}

\paragraph{Role}

\begin{itemize}
    \item název role
    \begin{itemize}
        \item popis
    \end{itemize}
\end{itemize}

\paragraph{Datové objekty}

\begin{itemize}
    \item název datového objektu
    \begin{itemize}
        \item popis [states]
    \end{itemize}
\end{itemize}

\paragraph{Diagram}

[zde bude BPMN diagram]

\subsection{Proces \uv{Příprava materiálů k receptům (fotky, ...).}}

\paragraph{Popis}

[text]

\paragraph{Indikátory}

\begin{itemize}
    \item název indikátoru
    \begin{itemize}
        \item units [formula]
        \item desired value [minimal value]
    \end{itemize}
\end{itemize}

\paragraph{Role}

\begin{itemize}
    \item název role
    \begin{itemize}
        \item popis
    \end{itemize}
\end{itemize}

\paragraph{Datové objekty}

\begin{itemize}
    \item název datového objektu
    \begin{itemize}
        \item popis [states]
    \end{itemize}
\end{itemize}

\paragraph{Diagram}

[zde bude BPMN diagram]

\subsection{Proces \uv{Plánování tras.}}

\paragraph{Popis}

[text]

\paragraph{Indikátory}

\begin{itemize}
    \item název indikátoru
    \begin{itemize}
        \item units [formula]
        \item desired value [minimal value]
    \end{itemize}
\end{itemize}

\paragraph{Role}

\begin{itemize}
    \item název role
    \begin{itemize}
        \item popis
    \end{itemize}
\end{itemize}

\paragraph{Datové objekty}

\begin{itemize}
    \item název datového objektu
    \begin{itemize}
        \item popis [states]
    \end{itemize}
\end{itemize}

\paragraph{Diagram}

[zde bude BPMN diagram]

\section{Implementace}

[text]

\subsection{Použitá platforma a software}

[text]

\subsection{Implementované služby}

[text]

\begin{enumerate}
    \item název služby
    \begin{itemize}
        \item text
    \end{itemize}
\end{enumerate}

\subsection{Implementované procesy}

\begin{enumerate}
    \item název procesu
    \begin{itemize}
        \item text
    \end{itemize}
\end{enumerate}

\subsection{Screenshoty}

[3-5 důležitých snímků z testování implementace]

\section{Týmová práce a úkoly}

\subsection{Tadeáš Pavlík (Teamleader)}

\begin{itemize}
    \item úkol 1
\end{itemize}

\subsection{Jiří Čechák (Business analyst)}

\begin{itemize}
    \item úkol 1
\end{itemize}

\subsection{Tomáš Došlík (Process analyst)}

\begin{itemize}
    \item úkol 1
\end{itemize}

\subsection{Václav Stehlík (BPM/SOA developer)}

\begin{itemize}
    \item úkol 1
\end{itemize}

\end{document}
